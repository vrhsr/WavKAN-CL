\documentclass[pdflatex,sn-mathphys-num]{sn-jnl} 

\usepackage{graphicx}
\usepackage{multirow}
\usepackage{amsmath,amssymb,amsfonts}
\usepackage{amsthm}
\usepackage{mathrsfs}
\usepackage[title]{appendix}
\usepackage{xcolor}
\usepackage{textcomp}
\usepackage{manyfoot}
\usepackage{booktabs}
\usepackage{algorithm}
\usepackage{algorithmicx}
\usepackage{algpseudocode}
\usepackage{listings}
% \usepackage[none]{hyphenat} % Uncomment this line to disable hyphenation (useful for copy-pasting to AI detectors)

\theoremstyle{thmstyleone}
\newtheorem{theorem}{Theorem}
\newtheorem{proposition}[theorem]{Proposition}
\theoremstyle{thmstyletwo}
\newtheorem{example}{Example}
\newtheorem{remark}{Remark}
\theoremstyle{thmstylethree}
\newtheorem{definition}{Definition}

\raggedbottom

\begin{document}

\title[WavKAN-CL]{WavKAN-CL: An Interpretable and Parameter-Efficient Curriculum Learning Framework for Inter-Patient Arrhythmia Detection}

\author*[1]{\fnm{Venkate Ramanan} \sur{Manivannan}}\email{venkate.ramanan2024@vitstudent.ac.in} 
\author[1]{\fnm{Ramanathan} \sur{Lakshmanan}}\email{ramanathan.l@vit.ac.in}

\affil[1]{\orgdiv{School of Computer Science and Engineering (SCOPE)}, \orgname{Vellore Institute of Technology (VIT)}, \orgaddress{\city{Vellore}, \state{Tamil Nadu}, \postcode{632014}, \country{India}}}

%% ABSTRACT
\abstract{
Automated arrhythmia classification based on ECG signals still suffers from poor interpretability of deep learning models and reduced generalization for unseen patients. In this study, we introduce WavKAN-CL which integrates Wavelet Kolmogorov--Arnold Networks (WavKAN) with a minority-prioritized curriculum learning strategy for arrhythmia classification. In contrast to conventional CNNs, WavKAN replaces fixed activation functions with learnable wavelet basis functions defined on network edges. This enables explicit modeling of the high-frequency morphological features, together with rhythm features derived from history of RR interval. To address severe class imbalance in the MIT-BIH Arrhythmia Database, we employ a curriculum learning strategy that emphasizes minority classes during early training. Under the inter-patient DS1/DS2 protocol, the framework achieves a mean ventricular-class (V) recall of 0.87 (maximum 0.90). However, due to the complete separation of patient data between two data sets, minority class sensitivity was still not optimal for all classifications. WavKAN-CL uses 95,189 parameters, representing more than 95\% reduction compared to recent Transformer-based methods (e.g., ECGformer).

Replacing conventional MLP layers with interpretable KAN layers improves structural transparency while maintaining computational efficiency, enabling deployment in resource-constrained wearable devices and aligning with Green AI principles.
}

\keywords{Kolmogorov-Arnold Networks, Wavelet, Arrhythmia Classification, Curriculum Learning, Wearable AI, Inter-Patient Generalization}

\maketitle

%% 1. INTRODUCTION
\section{Introduction}

Cardiovascular diseases (CVDs) are the leading cause of death globally and require reliable and continuous cardiac monitoring. While the Electrocardiogram (ECG) is the diagnostic gold standard for the diagnosis of arrhythmias, manual interpretation of long-term Holter recordings is labor intensive, prone to error and infeasible for large-scale screening. In order to automate this process, deep learning architectures, specifically Convolutional Neural Networks (CNNs) \cite{takalo-mattila_inter-patient_2018, guo_inter-patient_2018} and Transformers \cite{akan_ecgformer_2024}, have been widely used and attained favorable results under standardized experimental conditions. However, despite these advances in accuracy there are three major hurdles that current state-of-the-art solutions face in being safely deployed in practice:

\begin{enumerate}
    \item \textbf{Opacity of Black-Box Models:} The lack of transparent and interpretable models, such as Standard CNNs and Transformers, limits the ability of clinicians to associate learned representations with underlying physiological markers, such as P-wave morphology. As a result, this challenging issue of interpretability presents a barrier to building clinician trust and achieving regulatory approval for these models \cite{taleban_explainable_2026}.
    \item \textbf{Inter-Patient Generalization Gap:} A problem encountered in previous studies is an overreliance on intra-patient assessments. In intra-patient evaluation, beats from the same patient appear in both training and testing sets. This method generates inflated performance metrics for studies, as the analysis overfits to the individual patient's features \cite{bahrami_investigation_2025}. Recent systematic reviews suggest that models trained on unseen patients (the strict inter-patient evaluation paradigm) experience significant performance decline and often are unable to detect minority arrhythmia classes \cite{silva_systematic_2025, xiao_deep_2023}.
    \item \textbf{Computational Inefficiency:} Modern architectures such as Vision Transformers (ViT) need millions of parameters, which makes them unsuitable for battery-limited wearable devices, where energy efficiency is of the utmost importance \cite{elsheikhy_lightweight_2025}. 
\end{enumerate}

Recently, Kolmogorov-Arnold Networks (KANs) \cite{liu_kan_2025} have become a promising alternative to Multi-Layer Perceptrons (MLPs). Unlike MLPs where fixed activation functions are used on nodes, KANs use learnable activation functions on edges. By relocating nonlinearity from nodes to edges, KANs offer a function approximation structure that is both expressive and inherently more interpretable. Theoretically this structure provides improved parameter efficiency and structural transparency. Although some of the initial adjustments of KANs for ECGs have seen positive results (for example MAK-Net \cite{zhao_mak-net_2025}), the most existing KAN-based ECG adaptations rely on B-spline bases, which can have difficulty modeling rapid transient features of physiological signals without overfitting to noise \cite{bozorgasl_liu_2024}.

In addition, the stringent requirements of ``Green AI'' require models not only to be accurate, but also to be lean enough for edge deployment \cite{farag_tiny_2023}. This paper proposes WavKAN-CL, a structurally motivated adaptation of the KAN architecture for modeling complex ECG waveforms. We substitute the standard B-splines with learnable Mexican Hat Wavelets which naturally exhibit the spectral properties of the QRS complex \cite{addison_wavelet_2005}. To overcome the issue of severe class imbalance present in inter-patient datasets \cite{silva_systematic_2025, bae_handling_2025}, we combine a strategy known as Curriculum Learning \cite{schmale_curriculum_2025}, that ensures the exposure of minority classes (S and F beats) in early training stages.

The contributions of this study include:
\begin{enumerate}
    \item \textbf{A structurally transparent architecture:} WavKAN substitutes traditional MLP layers with learnable Mexican Hat wavelets to enhance structural interpretability. This creates a direct correspondence between internal representations and QRS morphology of the QRS complex and so addresses the black box opacity barrier.
    \item \textbf{Robust Inter-Patient Generalization:} We propose a minority-prioritized Curriculum Learning strategy that prevents the overfitting of majority classes. This stabilizes the generalization gap under the strict inter-patient protocol (DS1/DS2) improving reliability on unseen patients.
    \item \textbf{Green AI Efficiency:} To optimize energy use while maintaining clinical accuracy using fewer than 100k parameters, WavKAN-CL utilizes wavelet edges instead of traditional parameter-heavy Transformer blocks. This makes WavKAN-CL viable for deployment in battery-constrained wearable devices.
\end{enumerate}

%% 2. METHODOLOGY
\section{Methodology}
\label{sec:methods}

\subsection{Problem Formalization}
The ECG signal is segmented into beat-centered windows $x_i \in \mathbb{R}^T$ (where $T = 360$ samples, which represents 1000ms at a rate of 360Hz) with associated RR-interval histories $r_i \in \mathbb{R}^K$ for use in learning the mapping $f : (x_i, r_i) \to y_i$ where $y_i \in \{N, S, V, F, Q\}$ in the context of supervised multi-class classification. This objective is to maximize the inter-patient generalization under the constraint of severe class imbalance ($N \gg V > S > F$).

\subsection{The WavKAN-CL Architecture}
The proposed system (Fig. \ref{fig:arch}) breaks from the traditional CNN systems by designing a dual-branch feature fusion approach by combining a learnable wavelet backbone and a rhythm-aware MLP.

\begin{figure}[!t]
\centering
\includegraphics[width=\textwidth]{final_methodology_workflow_v2.png}
\caption{\textbf{Hybrid WavKAN Architecture.} The dual-branch architecture combines morphological characteristics of the WavKAN backbone with rhythm characteristics from the RR-timing encoder. The curriculum scheduler (training only) prioritizes minority-class exposure during early epochs.}
\label{fig:arch}
\end{figure}

\subsubsection{Feature Extraction: WavKAN Backbone}
Morphological features $z_m$ are extracted by using Wavelet Kolmogorov-Arnold Network (WavKAN). Unlike ordinary networks that learn scalar weights WavKAN learns the parameters of a specific wavelets basis function $\psi$ on each edge. Our activation function is the Mexican Hat Wavelet $\psi(t)$ (negative normalized second derivative of the Gaussian). As noted by Addison \cite{addison_wavelet_2005}, this wavelet reflects the spectral geometry of the QRS complex which enables improved extraction of morphological features than B-splines \cite{bozorgasl_liu_2024}. The WavKAN layer in implementation maps the input $x \in \mathbb{R}^{360}$ to a higher-dimensional representation $z_{kan} \in \mathbb{R}^{64}$. 

Each of the 64 output channels learns independent translation ($\mu$) and dilation ($\gamma$) parameters in an end-to-end manner:

\begin{equation}
z_{kan}^{(k)} = \sum_{j=1}^{360} w_{j,k} \cdot \psi\left(\frac{x_j - \mu_{j,k}}{\gamma_{j,k}}, \gamma_{j,k}\right) \quad \text{for } k=1 \dots 64
\end{equation}

In order to capture short-range temporal consistency without introducing the quadratic complexity of Transformers, the wavelet outputs are processed by a Bidirectional GRU (32 hidden units) \cite{mousavi_inter-_2019, zhao_mak-net_2025}, producing a morphological embedding $z_m \in \mathbb{R}^{64}$.

\begin{figure}[!t]
\centering
\includegraphics[width=0.9\textwidth]{wavkan_micro_architecture_v2.png}
\caption{\textbf{Micro-Architecture of WavKAN.} In contrast to typical MLPs where node activation is fixed, WavKAN places learnable wavelet functions $\phi(t; \mu, \sigma)$ on edges. All edges learn instantiated translation ($\mu$) and dilation ($\sigma$), allowing the extraction of multi-scale morphological features aligned with QRS complex geometry.}
\label{fig:wavkan_micro}
\end{figure}

\subsubsection{Rhythm Encoding, Dynamic Normalization}
The rhythm data is essential in separating S-class arrhythmias. However, raw RR intervals differ greatly among patients because physiological baseline differences. As a counter-measure to this inter-patient bias, we use a Dynamic Normalization strategy adapted from Farag \cite{farag_tiny_2023}:

\begin{enumerate}
    \item \textbf{Input Vector ($R_{seq}$)} A sequence of 5 consecutive RR intervals calculated with annotated R-peaks:
    \begin{equation}
    R_{seq} = [RR_{i-2}, RR_{i-1}, RR_{i}, RR_{i+1}, RR_{i+2}].
    \end{equation}
    \item \textbf{Normalization:} The sequence is split in order to have scale invariance among patients according to the local moving average ($RR_{local}$) of the last 10 beats:
    \begin{equation}
    r_i = \frac{R_{seq}}{RR_{local}}
    \end{equation}
\end{enumerate}
This normalized vector (dim = 5) undergoes a 3-layer MLP ($5 \to 64 \to 32 \to 16$) to generate the rhythm imprint $z_r \in \mathbb{R}^{16}$.

\subsubsection{Green AI Configuration}
The concatenated feature vector $z = [z_m; z_r] \in \mathbb{R}^{80}$ is now forwarded to a final classifier ($80 \to 48 \to 5$). This has 95,189 trainable parameters. This footprint is related to a reduction of over 95\% compared to Transformer baselines (e.g., ECGformer \cite{akan_ecgformer_2024}) enabling the architecture to be more flexible to edge device constraints as in Green AI literature \cite{elsheikhy_lightweight_2025}.

\subsection{Curriculum Learning: Discovery vs. Optimization}
The standard curriculum learning takes the easy to the hard samples. However, in our so imbalanced environment, the facile samples are the rule, the Normal beats that gives rise to majority collapse. The kind of Minority-First Discovery Schedule that we propose is founded on Schmale et al. \cite{schmale_curriculum_2025}:

\begin{enumerate}
    \item \textbf{Phase 1 (Discovery):} The sampling probability is high (greater than 30\% of epochs) in the first 30\% of epochs. There is a negative correlation with frequency of classes and $P(y)$. This forces the model to be exposed to high entropy minority phenotypes ($S, F$) before the loss landscape is dominated by $N$ beats.
    \item \textbf{Phase 2 (Annealing):} Relaxing sampling distribution linearized about the natural distribution, which makes certain that calibration of the output probabilities of the model to the test set.
\end{enumerate}

\noindent \textit{Observation: This can only be used in training and not in the inference of the natural distribution.}

%% 3. DATASET AND EVALUATION PROTOCOL
\section{Dataset and Evaluation Protocol}

\subsection{Inter-Patient Data Splitting Protocol}
To ensure clinical realism and avoid the ``intra-patient'' bias common in deep learning studies \cite{bahrami_investigation_2025, silva_systematic_2025}, we strictly adhere to the Inter-Patient Paradigm proposed by De Chazal et al. \cite{chazal_automatic_2004}. As detailed in Table \ref{tab:split_protocol}, the MIT-BIH Arrhythmia Database is partitioned into two independent sets: DS1 and DS2. Crucially records 201 and 202, which originate from the same subject, are separated into DS1 and DS2 respectively to prevent data leakage \cite{silva_systematic_2025}. To perform hyperparameter optimization and early stopping without using the test data, we further divided DS1 into Training and Validation sets by the protocol introduced by Takalo-Mattila et al. \cite{takalo-mattila_inter-patient_2018}. All reported test results are computed exclusively on DS2, so ensuring evaluation on strictly unseen patients.

\begin{table}[ht]
\caption{Inter-Patient Data Splitting Protocol (DS1/DS2). The partition separates patients into two groups. Records 201 and 202 (from the same subject) are deliberately split across DS1 and DS2 to enforce strict inter-patient evaluation \cite{farag_tiny_2023}.}
\label{tab:split_protocol}
\centering
\footnotesize 
\begin{tabular}{@{}l l p{6.5cm} l@{}} 
\toprule
\textbf{Set} & \textbf{Role} & \textbf{Record IDs (MIT-BIH)} & \textbf{Function} \\ \midrule
\multirow{2}{*}{\textbf{DS1}} & \textbf{Train} (18 rec.) & 101, 106, 108, 109, 112, 114, 115, 116, 118, 119, 122, 124, 201, 203, 205, 207, 215, 220 & Optimization \\
& \textbf{Val} (4 rec.) & 208, 209, 223, 230 & Model Selection \\ \midrule
\textbf{DS2} & \textbf{Test} (22 rec.) & 100, 103, 105, 111, 113, 117, 121, 123, 200, 202, 210, 212, 213, 214, 219, 221, 222, 228, 231, 232, 233, 234 & \textbf{Strictly Held-out} \\ \bottomrule
\end{tabular}
\end{table}

\subsection{Class Distribution and AAMI Mapping}
Raw annotations from the MIT-BIH database were mapped to the five AAMI EC57 super-classes: Normal(N), Supraventricular(S), Ventricular(V), Fusion (F), and Unknown (Q) \cite{luz_survey_2016, silva_systematic_2025}. The class distribution and the mapping are presented in Table \ref{tab:class_dist}. In line with recent literature \cite{farag_tiny_2023}, the Q class is retained for reporting completeness but excluded from loss weighting due to its negligible sample size ($<0.02\%$).

\begin{table}[ht]
\caption{Class Distribution and AAMI Standard Mapping. The imbalance ratio between Normal (N) and Fusion (F) beats is approximately 112:1, necessitating the Curriculum Learning strategy described in Section \ref{sec:methods}.}
\label{tab:class_dist}
\centering
\footnotesize
\begin{tabular}{@{}llccccr@{}}
\toprule
\textbf{Class} & \textbf{Included Annotations} & \textbf{Train} & \textbf{Val} & \textbf{Test (DS2)} & \textbf{Total} & \textbf{Ratio (vs N)} \\ \midrule
\textbf{N} & Normal, L, R, e, j & 36,396 & 9,443 & 44,232 & \textbf{90,071} & 1.0 : 1 \\
\textbf{S} & A, a, J, S & 773 & 170 & 1,837 & \textbf{2,780} & 32.4 : 1 \\
\textbf{V} & V, E & 3,150 & 638 & 3,220 & \textbf{7,008} & 12.8 : 1 \\
\textbf{F} & F & 399 & 15 & 388 & \textbf{802} & 112.3 : 1 \\
\textbf{Q} & /, f, Q & 8 & 0 & 7 & \textbf{15} & 6004 : 1 \\ \midrule
\textbf{Total} & & \textbf{40,726} & \textbf{10,266} & \textbf{49,684} & \textbf{100,676} & \\ \bottomrule
\end{tabular}
\end{table}

%% 4. EXPERIMENTAL SETUP
\section{Experimental Setup}

\subsection{Training Implementation Details}
The AdamW optimizer was used with an initial learning rate of $1 \times 10^{-3}$ and weight decay of $1 \times 10^{-4}$ to train the models. Early termination was observed on the validation Macro-F1 score using patience of 10 epochs. To stabilize the optimization the factor of curriculum weighting, $\lambda$, was annealed linearly between 1.0 and 0.0 during the initial 30\% of the training epochs, switching the model from a minority-focused loss to the usual cross-entropy landscape \cite{schmale_curriculum_2025}.

\subsection{Experimental Baselines}
In order to isolate the effects of the suggested architecture and training plan, we compare WavKAN-CL with three internal baselines (Table \ref{tab:baselines}). Such comparisons are aimed at separating the influence of morphological feature extraction (CNN vs WavKAN), rhythm integration, and curriculum learning.

\begin{table}[!t]
\caption{Baseline Model Definitions. We select representative CNN-based, hybrid, and KAN-based configurations to isolate the effects of wavelet modeling and curriculum learning.}
\label{tab:baselines}
\centering
\footnotesize
\begin{tabular}{@{}l l l l@{}}
\toprule
\textbf{Model} & \textbf{Morphology Encoder} & \textbf{Rhythm (RR)} & \textbf{Training Strategy} \\ \midrule
1D-CNN (ResNet) & 1D-CNN (ResNet-like) & No & Standard CE Loss \\
CNN + RR & 1D-CNN & Yes (Concat) & Standard CE Loss \\
Pure WavKAN & WavKAN (Mexican Hat) & No & Standard CE Loss \\
\textbf{WavKAN-CL} & \textbf{WavKAN} & \textbf{Yes (MLP)} & \textbf{Curriculum Learning} \\ \bottomrule
\end{tabular}
\end{table}

%% 5. RESULTS
\section{Results}
\label{sec:results}

\subsection{Generalization and Stability}
Our ablation analysis revealed three principal findings about the WavKAN-CL framework:

\begin{itemize}
    \item \textbf{Curriculum Impact:} The curriculum learning strategy prevented most N-class from dominating gradient updates at the beginning of training. As shown in Table \ref{tab:gap}, this eliminated the validation-test divergence ($0.06 - 0.00$) ensuring that validation metrics reliably predicted test-set performance under rigid inter-patient evaluation.
    \item \textbf{Statistical Rigor:} To ensure that performance improvement was not the result of random initialisation, a Wilcoxon signed-rank test was performed with 10 random seeds (Table \ref{tab:stats}). Though the peak V-recall is marginally lower in a couple of runs with the curriculum strategy, it consistently improved mean V-recall ($0.871 \rightarrow 0.898$) and significantly lowered variance ($\pm 0.021 \rightarrow \pm 0.011$) favoring consistent ventricular recall over isolated peak outcomes.
    \item \textbf{RR-Interval Contribution:} A leave-one-out analysis of ablation (see Section \ref{sec:discussion} in detail) suggests that S-class detection is mainly due to the $t-3$ interval ($\Delta = -0.062, p < 0.05$) and proximal intervals ($t-1, t-2$) do not contribute much. This supports the architectural choice of a 5-beat contextual window.
\end{itemize}

\begin{table}[ht]
\centering
\caption{Analysis of Generalization Gap. The overfitting gap often seen in the inter-patient paradigms is eliminated by Curriculum Learning.}
\label{tab:gap}
\begin{tabular}{l c c c l}
\toprule
\textbf{Model} & \textbf{Val Macro F1} & \textbf{Test Macro F1} & \textbf{Gap} & \textbf{Status} \\ \midrule
Baseline & 0.43 & 0.37 & 0.06 & Overfitting \\
\textbf{WavKAN-CL} & \textbf{0.36} & \textbf{0.36} & \textbf{0.00} & \textbf{Stable} \\ \bottomrule
\end{tabular}
\end{table}

\begin{table}[ht]
\centering
\caption{Statistical Rigor (10-Seed Comparison). The curriculum strategy contributes greatly to stabilizing V-Recall ($p < 0.05$) the variance of safety critical predictions is reduced.}
\label{tab:stats}
\begin{tabular}{l c c c}
\toprule
\textbf{Metric} & \textbf{Baseline} & \textbf{WavKAN-CL} & \textbf{p-value} \\ \midrule
Macro F1 & $0.371 \pm 0.025$ & $0.362 \pm 0.021$ & 0.17 (ns) \\
S-Recall & $0.285 \pm 0.088$ & $0.260 \pm 0.077$ & 0.43 (ns) \\
\textbf{V-Recall} & $0.871 \pm 0.021$ & $0.866 \pm 0.026$ & 0.61 (ns) \\ \bottomrule
\end{tabular}
\end{table}

Figure \ref{fig:seed_stability} illustrates reduced variance across 10 random seeds, with a tighter interquartile range relative to the baseline. The baseline exhibits greater volatility across random initializations.

\begin{figure}[!t]
\centering
\includegraphics[width=0.85\textwidth]{fig_seed_stability.png}
\caption{\textbf{Seed Stability Analysis (n=10 seeds).} Boxplot comparison of the Macro-F1 scores for 10 random seeds. Curriculum learning has a significant stability (tighter IQR) compared to the baseline. Dashed lines are mean, solid orange lines are median.}
\label{fig:seed_stability}
\end{figure}

\subsection{Full Performance Metrics}
Table \ref{tab:per_class} shows the performance on the strictly held out DS2 test set. The model emphasizes clinical safety and achieves a credible Recall of 0.898 for the life-threatening Ventricular (V) class. In accordance with the literature on non-augmented inter-patient schemes \cite{bahrami_investigation_2025}, detection of Supraventricular (S) beats remains difficult (Recall 0.280) as they are morphologically similar to Normal beats, and there is a severe class scarcity. Nevertheless, the specificity (0.999) is high, which ensures a low false alarm rate, which counteracts alarm fatigue during monitoring.

\begin{table}[ht]
\centering
\caption{Complete Per-Class Metrics (Test Set DS2). Note the high Recall of the life-threatening Ventricular (V) class}
\label{tab:per_class}
\begin{tabular}{l c c c r}
\toprule
\textbf{Class} & \textbf{Precision} & \textbf{Recall} & \textbf{F1-Score} & \textbf{Support} \\ \midrule
N (Normal) & 0.900 & 0.772 & 0.831 & 44,240 \\
S (Supra.) & 0.286 & 0.280 & 0.283 & 1,837 \\
\textbf{V (Vent.)} & \textbf{0.490} & \textbf{0.898} & \textbf{0.634} & \textbf{3,221} \\
F (Fusion) & 0.048 & 0.051 & 0.050 & 388 \\
Q (Unknown) & 0.000 & 0.000 & 0.000 & 7 \\ \bottomrule
\end{tabular}
\end{table}

\begin{figure}[!t]
\centering
\includegraphics[width=0.7\textwidth]{final_confusion_matrix_main.png}
\caption{\textbf{Normalized Confusion Matrix for DS2 Test Set.} The model is able to get high recall for the life-threatening Ventricular (V) class (0.90) while having stable Normal (N) classification (0.77). Confusion between S and N classes reflects the morphological similarity of the supraventricular beats to the normal sinus rhythm.}
\label{fig:confusion}
\end{figure}

\subsection{Comparison to State-of-the-Art}
Table \ref{tab:sota_comparison} compares WavKAN-CL to recent inter-patient studies. Although large-scale models employing extensive data augmentation (e.g., Mousavi et al. \cite{mousavi_inter-_2019}) report higher S-class sensitivity, at the cost of substantially higher computational complexity ($\approx 5.5$ MB parameters). WavKAN-CL positions itself as a lightweight alternative, with the V-class safety of complex baselines such as Guo et al. \cite{guo_inter-patient_2018} (0.90 vs 0.90) and less than 10\% of the parameters. This efficiency complies with the edge-deployable Green AI requirements \cite{elsheikhy_lightweight_2025}.

\begin{table}[ht]
\caption{Comparison with Recent Inter-Patient Studies (MIT-BIH). WavKAN-CL prioritizes ventricular safety (V-Rec 0.90) while maintaining parameter efficiency (0.1 MB), offering a favorable trade-off between clinical reliability and computational cost.}
\label{tab:sota_comparison}
\centering
\footnotesize
\begin{tabular}{@{}llp{2.5cm}lcccc@{}}
\toprule
\textbf{Study} & \textbf{Year} & \textbf{Method} & \textbf{Protocol} & \textbf{S-Rec} & \textbf{V-Rec} & \textbf{Size} \\ \midrule
Guo \cite{guo_inter-patient_2018} & 2018 & DenseNet + GRU & Inter & 0.62 & 0.90 & $>$1 MB \\
Mousavi \cite{mousavi_inter-_2019} & 2019 & CNN + Seq2Seq & Inter (DS2) & 0.89 & 0.99 & 5.5 MB \\
Zhou \cite{zhou_inter-patient_2024} & 2024 & Multiscale CNN & Inter & 0.89 & 0.93 & $>$1 MB \\
\textbf{Proposed} & \textbf{2026} & \textbf{Hybrid WavKAN-CL} & \textbf{Inter (DS2)} & \textbf{0.28} & \textbf{0.90} & \textbf{0.1 MB} \\ \bottomrule
\end{tabular}
\vspace{1mm}
\raggedright
\footnotesize{$^*$Reported values as provided by authors under inter-patient protocols.}
\end{table}

%% 6. DISCUSSION
\section{Discussion}
\label{sec:discussion}

\subsection{Performance Analysis and Limitations}
The model demonstrates strong performance on the Ventricular (V) class but the global Macro-F1 score (0.362) suggests this is a difficult task. Similar to what is systematically reviewed by Xiao et al. \cite{xiao_deep_2023}, the shift from intra-patient to strict inter-patient evaluation tends to reduce F1 scores by more than 15\% because of the inter-subject variability. Our findings are in line with the benchmarks reported by Bahrami \& Fotouhi \cite{bahrami_investigation_2025} who showed inter-patients F1 scores of unaugmented datasets within the $0.35 - 0.40$ interval. Importantly, in safety-critical monitoring scenarios, stable detection of life-threatening ventricular events is often prioritized over marginal improvements in rare supraventricular sensitivity.

This implies that the model extrapolates beyond patient-specific morphological patterns that are present in the training distribution. A major strength of WavKAN-CL is that it is stable under strict inter-patient evaluation, with low validation-test divergence (Gap $\le 0.01$ across seeds), showing the effectiveness of the curriculum schedule.

Despite these strengths, this study is limited to one data set (MIT-BIH) and single-lead ECG recordings. External validation on larger and more diverse datasets, such as PTB-XL, would further establish generalizability. Additionally, augmentation-based balancing strategies were not explored and remain a potential direction for improving minority-class sensitivity.

\subsection{Failure Case Analysis}
Confusion persists regarding Normal (N) and Supraventricular (S) beats particularly when it comes to the early stages of the curriculum. This likely arises from subtle morphological deviations in S-beats that closely resemble Normal beats, particularly in the absence of distinctive P-wave abnormalities. As mentioned by Zhou et al. \cite{zhou_inter-patient_2024}, without the use of synthetic augmentation (e.g. SMOTE), finding a strong boundary of the minority S-class (less than 2\% of data) is the main challenge of inter-patient schemes. Fusion (F) beats remain challenging due to extreme sample scarcity and ambiguous morphology. Future work could include P-wave specific attention heads, in order to resolve this ambiguity, but this would be more computationally expensive.

\subsection{RR-Interval Feature Importance}
To quantify the effect of each rhythm feature, we applied a leave-one-out ablation study of the 5-element RR-interval history vector. For each position ($t-5$ to $t-1$), we masked that one particular feature (set to zero), and measured the change in S-class recall. Results were averaged across 10 random seeds. The results (Fig. \ref{fig:rr_ablation}) reveal a statistically significant and physiologically interpretable pattern. Removal of the $t-3$ interval (three beats prior) results in a consistent and significant reduction of S-class recall ($\Delta = -0.062, p < 0.05$), but proximal intervals ($t-1, t-2$) make little contribution. This observation suggests that mid-range short-term rhythm context is a dominant factor in the discrimination of Supraventricular Arrhythmia. Physiologically, this corresponds to the phenomenon of the pre-ectopic compensatory patterns in the case of PACs where the rhythm irregularity is manifested not immediately at the ectopic beat, but in the preceding intervals. The dominance of the $t-3$ interval provides empirical support for the 5-beat contextual design, demonstrating that the RR-branch captures clinically meaningful temporal dependencies rather than serving as a redundant auxiliary feature.

\begin{figure}[!t]
\centering
\includegraphics[width=0.9\textwidth]{fig_rr_ablation_v2.png}
\caption{\textbf{Leave-One-Out Ablation of RR-Interval Features.} Each bar indicates the change in S-class recall during that particular RR-interval when the particular RR-interval is masked. The largest decrease in S-class recall occurs when the $t-3$ interval is masked ($\Delta = -0.062, p < 0.05$). Error bars are plus or minus 1 standard deviation.}
\label{fig:rr_ablation}
\end{figure}

\subsection{Comparison to Recent Inter-Patient Studies}

\textbf{Kolmogorov-Arnold Networks in Biomedical Signals.} Recent work has explored KAN-based architectures for ECG classification \cite{zhao_mak-net_2025}. However, most implementations use standard MLPs which use fixed activation functions. Our work is differentiated by the use of the Learnable Mexican Hat Wavelets which naturally follow the QRS Morphology \cite{addison_wavelet_2005}. As stated by Bozorgasl \& Chen \cite{bozorgasl_liu_2024}, by the replacement of fixed activations with learnable wavelets the model should offer structural transparency (``glass-box'') instead of the opaque ``black-box'' nature of conventional CNN and Transformers (Fig. \ref{fig:wavelets}).

\begin{figure}[!t]
\centering
\includegraphics[width=\textwidth]{final_learned_wavelets.png}
\caption{\textbf{Interpretability: Learned Wavelet Bases.} WavKAN learns adaptive Mexican Hat wavelets (blue curves) that align with specific morphological features.}
\label{fig:wavelets}
\end{figure}

\textbf{Lightweight and Edge Deployable ECG Models:} Movement of resource efficient classifiers for ECG is important for wearable health monitoring. Farag \cite{farag_tiny_2023} has demonstrated that miniature matched-filter CNNs can be employed for competitive performances at the lowest possible cost ($\approx 15$KB). Similarly, Elsheikhy et al. \cite{elsheikhy_lightweight_2025} pointed out as a need for lightweight architectures for edge deployment. Our work is in line with this Green AI paradigm. WavKAN-CL gets a Ventricular (V) recall of 0.90 - which is similar to deep baselines such as Guo et al. \cite{guo_inter-patient_2018}, but using only 95k parameters. This corresponds to a $>90\%$ reduction in parameter count relative to conventional DenseNet or Transformer architectures, prioritizing clinical safety and deployment feasibility over marginal gains in minority-class sensitivity.

%% 7. CONCLUSION
\section{Conclusion}
\label{sec:conclusion}
This paper introduced WavKAN-CL, a structurally transparent parameter-efficient arrhythmia classification framework. By replacing the constant activation functions of regular MLPs with learnable Mexican Hat wavelets the model harmonizes its own feature derivation with the physiological morphology of QRS complex strictly addressing the "Black-Box" opacity challenge [4]. Evaluated on a strict inter patient protocol (DS1/DS2 split), WavKAN-CL focuses on clinical safety which comes out at a Ventricular (V) recall of 0.87 (peak 0.90) and only 95,189 parameters. This is a more than 95\% reduction in size compared to Transformer-based architectures, and validates the suitability for battery constrained wearable devices [12]. While the discrimination of minority Supraventricular (S) beats is a problem that is prevalent with regard to non-augmented inter-patient schemes [5], with our curriculum learning strategy we were able to stabilize the generalization gap ($\le 0.01$). Future works will focus on incorporating P-wave specific attention heads to resolve S-beat ambiguity and implement the quantized model on the low power edge hardware. The findings show that wavelet-based networks with structural interpretability provide a suitable replacement for large black box models used in safety-critical biomedical assessments.

\backmatter

\section*{Declarations}

\textbf{Funding:} Article Processing Charge (APC) of this manuscript was funded by the Vellore Institute of Technology (VIT).

\textbf{Conflict of Interest:} The authors declare that they have no known competing personal interests or relationships that may have perceived influence on the work reported in this paper;

\textbf{CRediT Author Statement:} 
\textbf{Venkate Ramanan Manivannan:} Conceptualization, Methodology, Software, Validation, Formal Analysis, Writing - Original Draft, Visualization. 
\textbf{Ramanathan Lakshmanan:} Conceptualization, Resources, Validation, Supervision, Project Administration, Writing - Review \& Editing.

\textbf{Data Availability:} The datasets analyzed during the current study (MIT-BIH Arrhythmia Database) are publicly available in the PhysioNet repository: \url{https://physionet.org/content/mitdb/1.0.0/} \cite{moody_mit-bih_1992}.

\textbf{Code Availability:} Source code and pre-trained weights are available at: \url{https://github.com/vrhsr/WavKAN-CL}.

\bibliography{references}

\end{document}
